\documentclass[17pt,aspectratio=169]{beamer}

\usepackage[T1,T2A]{fontenc}
\usepackage[utf8]{inputenc}
\usepackage[english,russian]{babel}
\usepackage{textcomp}
\usepackage[normalem]{ulem}
\usepackage{hyperref} 

\usetheme{Madrid}

\usepackage{graphicx}

\usepackage{xcolor}
\definecolor{bluekeywords}{rgb}{0,0,1}
\definecolor{greencomments}{rgb}{0,0.5,0}
\definecolor{redstrings}{rgb}{0.64,0.08,0.08}
\definecolor{xmlcomments}{rgb}{0.5,0.5,0.5}
\definecolor{types}{rgb}{0.17,0.57,0.68}

\usepackage{listings}
\lstset{
language=[Sharp]C, 
showspaces=false,
showtabs=false,
breaklines=true,
showstringspaces=false,
breakatwhitespace=true,
escapeinside={(*@}{@*)},
morekeywords={partial, var, value, get, set, async, await, Task},
commentstyle=\color{greencomments},
keywordstyle=\color{bluekeywords},
stringstyle=\color{redstrings},
basicstyle=\ttfamily\small
}

\AtBeginSection[]{
  \begin{frame}
  \vfill
  \centering
  \begin{beamercolorbox}[sep=8pt,center,shadow=true,rounded=true]{title}
    \usebeamerfont{title}\insertsectionhead\par%
  \end{beamercolorbox}
  \vfill
  \end{frame}
}

\title{HttpClient} 
\subtitle{прошлое, настоящее, будущее}
\author{Риваль Абдрахманов}
\institute[PT]{Positive Technologies} 
\date{SpbDotNet, 2019}

\begin{document}
\begin{frame}
\titlepage
\end{frame}

\begin{frame}
\frametitle{Содержание}
\tableofcontents
\end{frame}

\section{HttpClient - базовая информация}

\begin{frame}{\href{https://docs.microsoft.com/en-us/dotnet/api/system.net.http.httpclient?view=netcore-2.2}{HttpClient Class}}
    \begin{itemize}
        \item <1-> Базовый класс для отправки HTTP-запросов и получения HTTP-ответов;
        \item <2-> $GetAsync(\ldots)$, $PostAsync(\ldots)$, $SendAsync(\ldots)$ и др.;
        \item <3-> $HttpClient$ реализует $IDisposable$.
    \end{itemize}
\end{frame}

\begin{frame}{\href{https://docs.microsoft.com/en-us/dotnet/api/system.idisposable?view=netcore-2.2}{IDisposable Interface}}
    \begin{itemize}
        \item <1-> Предоставляет механизм для освобождения неуправляемых ресурсов;
        \item <2-> $public\,void\,Dispose()$;
        \item <3-> Конструкция $using(\ldots)$;
        \item <4-> Диспозиться\,\textrightarrow \,диспозь
    \end{itemize}
\end{frame}

\begin{frame}{\href{https://docs.microsoft.com/en-us/dotnet/api/system.idisposable?view=netcore-2.2}{IDisposable Interface}}
    \begin{itemize}
        \item Provides a mechanism for releasing unmanaged resources;
        \item $public\,void\,Dispose()$;
        \item Конструкция $using(\ldots)$;
        \item \sout{Диспозиться\,\textrightarrow \,диспозь}
        \item Диспозиться\,\textrightarrow \,будь внимательней
    \end{itemize}
\end{frame}

\begin{frame}[fragile]
\frametitle{Disposable HttpClient}
Мотивация:
\newline
\begin{lstlisting}
using(var client = new HttpClient())
{
  var response = 
    await client.GetStringAsync(...);
}
\end{lstlisting}
\end{frame}

\section{Неочевидные проблемы}
\begin{frame}
\frametitle{Проблема socket exhaustion}
\href{https://aspnetmonsters.com/2016/08/2016-08-27-httpclientwrong/}{https://aspnetmonsters.com/2016/08/2016-08-27-httpclientwrong/}
\begin{figure}
\includegraphics[scale=0.3]{aspnetmonsters}
\end{figure}
\end{frame}

\begin{frame}[fragile]
\frametitle{Проблема socket exhaustion}
\begin{lstlisting}
for(int i = 0; i < 10; i++)
{
  using (var client = new HttpClient())
  {
    await client
      .GetStringAsync("https://google.com");
  }
}
\end{lstlisting}
\end{frame}

\begin{frame}
\frametitle{Проблема socket exhaustion}
Проверяем через netstat:
\begin{figure}
\includegraphics[scale=0.53]{netstat}
\end{figure}
\end{frame}

\begin{frame}
\frametitle{Проблема socket exhaustion}
\begin{itemize}
	\item 10 сокетов в состоянии \textit{TIME WAIT};
	\item Соединение закрыто с одной стороны, но ещё ждём доходящие пакеты;
	\item Приводит к $SocketException$.
\end{itemize}
\end{frame}

\begin{frame}
\frametitle{Проблема socket exhaustion}
``HttpClient предназначен для однократного создания экземпляра и повторного использования в течение всего жизненного цикла приложения.''
\newline
\newline
\textit{\href{https://docs.microsoft.com/en-us/dotnet/api/system.net.http.httpclient}{https://docs.microsoft.com/en-us/dotnet/api/system.net.http.httpclient}}
\end{frame}

\begin{frame}[fragile]
\frametitle{Проблема socket exhaustion}
Решение проблемы - переиспользование клиента:
\newline
\begin{lstlisting}
private static HttpClient Client 
  = new HttpClient();
\end{lstlisting}
\end{frame}

\begin{frame}
\frametitle{Проблема кеширования DNS}
\href{https://byterot.blogspot.com/2016/07/singleton-httpclient-dns.html}{https://byterot.blogspot.com/2016/07/singleton-httpclient-dns.html}
\begin{figure}
\includegraphics[scale=0.4]{byterot}
\end{figure}
\end{frame}

\begin{frame}
\frametitle{Проблема кеширования DNS}
\begin{itemize}
	\item Не учитываются изменения DNS;
	\item Соединение держится до закрытия сокета.
\end{itemize}
\end{frame}

\begin{frame}
\frametitle{Проблема кеширования DNS}
Решение для .NET Framework:
\begin{itemize}
	\item <1-> Класс $ServicePointManager$;
	\item <2-> $ServicePointManager.DnsRefreshTimeout$ (2 мин);
	\item <3-> $ServicePoint.ConnectionLeaseTimeout$ (не ограничено);
	\item <4-> $ServicePoint.MaxIdleTime$ (100 сек).
\end{itemize}
\end{frame}

\begin{frame}[fragile]
\frametitle{Проблема кеширования DNS}
Пример:
\newline
\begin{lstlisting}
ServicePointManager.DnsRefreshTimeout = 60000;

var sp = ServicePointManager
  .FindServicePoint(new Uri("http://site.com"));
sp.ConnectionLeaseTimeout = 60000;
sp.MaxIdleTime = 60000;
\end{lstlisting}
\end{frame}

\begin{frame}
\frametitle{Лимит одновременных соединений}
\href{https://habr.com/ru/post/424873/}{https://habr.com/ru/post/424873/}
\begin{figure}
\includegraphics[scale=0.4]{habr}
\end{figure}
\end{frame}

\begin{frame}
\frametitle{Лимит одновременных соединений}
\begin{itemize}
	\item <1-> Лимит по умолчанию равен 2;
	\item <2-> $ServicePointManager.DefaultConnectionLimit$;
	\item <3-> Для $localhost$ по умолчанию равен $int.MaxValue$;
	\item <4-> Только для .NET Framework.
\end{itemize}
\end{frame}

\begin{frame}
\frametitle{Выводы}
\begin{itemize}
	\item Нельзя на каждый запрос переоткрывать соединение;
	\item Нельзя постоянно держать соединение открытым;
	\item Вручную управлять сложно.
\end{itemize}
\end{frame}

\section{Интерфейс IHttpClientFactory}
\begin{frame}
\frametitle{\href{https://docs.microsoft.com/en-us/dotnet/api/system.net.http.ihttpclientfactory?view=aspnetcore-2.2}{IHttpClientFactory}}
\begin{itemize}
	\item <1-> Позволяет создавать и конфигурировать $HttpClient$;
	\item <2-> Был добавлен в ASP.NET Core 2.1;
	\item <3-> Для консольного приложения необходимо добавить \href{https://www.nuget.org/packages/Microsoft.Extensions.Hosting}{$Microsoft.Extensions.Hosting$} и \href{https://www.nuget.org/packages/Microsoft.Extensions.Http}{$Microsoft.Extensions.Http$}
\end{itemize}.
\end{frame}

\begin{frame}[fragile]
\frametitle{IHttpClientFactory}
Регистрация через метод расширения $IServiceCollection$:
\newline
\begin{lstlisting}
services.AddHttpClient();
\end{lstlisting}
\end{frame}

\begin{frame}[fragile]
\frametitle{IHttpClientFactory}
Добавление в конструктор с помощью DI:
\newline
\begin{lstlisting}
public SomeService(IHttpClientFactory clientFactory)
{
  _clientFactory = clientFactory;
}
\end{lstlisting}
\end{frame}

\begin{frame}[fragile]
\frametitle{IHttpClientFactory}
Создание клиента:
\newline
\begin{lstlisting}
var client = _clientFactory.CreateClient();
var response = await client.SendAsync(request);
\end{lstlisting}
\end{frame}

\begin{frame}[fragile]
\frametitle{Named clients}
Регистрация через метод расширения $IServiceCollection$:
\newline
\begin{lstlisting}
services.AddHttpClient("some-site", c =>
{
  c.BaseAddress = 
    new Uri("https://some-site.com/");
  c.DefaultRequestHeaders
    .Add("Accept", "application/json");
});
\end{lstlisting}
\end{frame}

\begin{frame}[fragile]
\frametitle{Named clients}
Добавление в конструктор с помощью DI:
\newline
\begin{lstlisting}
public SomeService(IHttpClientFactory clientFactory)
{
  _clientFactory = clientFactory;
}
\end{lstlisting}
\end{frame}

\begin{frame}[fragile]
\frametitle{Named clients}
Создание клиента:
\newline
\begin{lstlisting}
var client = _clientFactory.CreateClient("some-site");
\end{lstlisting}
\end{frame}

\begin{frame}[fragile]
\frametitle{Typed clients}
Класс типизированного клиента:
\begin{lstlisting}
public class SomeSiteClient
{
  ...
}
\end{lstlisting}
\end{frame}

\begin{frame}[fragile]
\frametitle{Typed clients}
Класс типизированного клиента:
\begin{lstlisting}
private readonly HttpClient _client;
public SomeSiteClient(HttpClient client)
{
  client.BaseAddress = new Uri("https://some-site.com/");
  client.DefaultRequestHeaders.Add("Accept", "application/json");

  _client = client;
}
\end{lstlisting}
\end{frame}

\begin{frame}[fragile]
\frametitle{Typed clients}
Класс типизированного клиента:
\begin{lstlisting}
public async Task<SomeData> GetSomeData()
{
  var response = 
    await _client.GetAsync("/get-some-data");
  ...
  return result;
}
\end{lstlisting}
\end{frame}

\begin{frame}[fragile]
\frametitle{Typed clients}
Регистрация типизированного клиента:
\newline
\begin{lstlisting}
services.AddHttpClient<SomeSiteClient>();
\end{lstlisting}
\end{frame}

\begin{frame}[fragile]
\frametitle{Typed clients}
Добавление в конструктор через DI:
\newline
\begin{lstlisting}
public SomeService(SomeSiteClient someSiteClient)
{
  _someSiteClient = someSiteClient;
}
\end{lstlisting}
\end{frame}

\begin{frame}[fragile]
\frametitle{\href{https://github.com/reactiveui/refit}{Refit}}
Библиотека Refit для REST API (\href{https://github.com/reactiveui/refit}{https://github.com/reactiveui/refit})
\end{frame}

\begin{frame}[fragile]
\frametitle{\href{https://github.com/reactiveui/refit}{Refit}}
\begin{lstlisting}
public interface ISomeSiteClient
{
  [Get("/get-some-data")]
  Task<SomeData> GetSomeData();
}
\end{lstlisting}
\end{frame}

\begin{frame}[fragile]
\frametitle{\href{https://github.com/reactiveui/refit}{Refit}}
Регистрация клиента:
\newline
\begin{lstlisting}
services
  .AddRefitClient<ISomeSiteClient>()
  .ConfigureHttpClient(c => c.BaseAddress = 
    new Uri("https://some-site.com"));
\end{lstlisting}
\end{frame}

\begin{frame}[fragile]
\frametitle{\href{https://github.com/reactiveui/refit}{Refit}}
Добавление в конструктор через DI:
\newline
\begin{lstlisting}
public SomeService(ISomeSiteClient someSiteClient)
{
  _someSiteClient = someSiteClient;
}
\end{lstlisting}
\end{frame}

\begin{frame}[fragile]
\frametitle{Создание HttpClient}
Посмотрим глубже, как происходит создание $HttpClient$
\newline
\begin{figure}
\includegraphics[scale=0.4]{dno}
\end{figure}
\end{frame}

\begin{frame}[fragile]
\frametitle{Создание HttpClient}
\begin{lstlisting}
public class HttpClient : HttpMessageInvoker
\end{lstlisting}
\begin{lstlisting}
public class HttpMessageInvoker : IDisposable
\end{lstlisting}
\end{frame}

\begin{frame}[fragile]
\frametitle{Конструкторы HttpMessageInvoker}
\begin{lstlisting}
public HttpMessageInvoker(HttpMessageHandler handler, bool disposeHandler)

public HttpMessageInvoker(HttpMessageHandler handler) : this(handler, true)
\end{lstlisting}
\end{frame}

\begin{frame}[fragile]
\frametitle{Конструкторы HttpClient}
\begin{lstlisting}
public HttpClient(HttpMessageHandler handler, bool disposeHandler) : base(handler, disposeHandler)

public HttpClient(HttpMessageHandler handler) : this(handler, true)

public HttpClient() : this((HttpMessageHandler) new HttpClientHandler())
\end{lstlisting}
\end{frame}

\begin{frame}
\frametitle{Конструкторы HttpClient}
\begin{itemize}
	\item 3 конструктора;
	\item Есть возможность передать $HttpMessageHandler$;
	\item В стандартном конструкторе $disposeHandler$ равен $true$.
\end{itemize}
\end{frame}

\begin{frame}[fragile]
\frametitle{Dispose HttpClient}
\begin{lstlisting}
protected override void Dispose(bool disposing)
{
  if (disposing && !this._disposed)
  {
    this._disposed = true;
    this._pendingRequestsCts.Cancel();
    this._pendingRequestsCts.Dispose();
  }
  base.Dispose(disposing);
}
\end{lstlisting}
\end{frame}

\begin{frame}[fragile]
\frametitle{Dispose HttpMessageInvoker}
\begin{lstlisting}
public void Dispose()
{
  this.Dispose(true);
  GC.SuppressFinalize((object) this);
}
\end{lstlisting}
\end{frame}

\begin{frame}[fragile]
\frametitle{Dispose HttpMessageInvoker}
\begin{lstlisting}
protected virtual void Dispose(bool disposing)
{
  if (!disposing || this._disposed)
    return;
  this._disposed = true;
  if (!this._disposeHandler)
    return;
  this._handler.Dispose();
}
\end{lstlisting}
\end{frame}

\begin{frame}
\frametitle{Dispose HttpClient}
\begin{itemize}
	\item Отменяются все повисшие запросы $\Rightarrow$ могут отменяться чужие запросы;
	\item Флаг $disposed$ выставляется в $true$;
	\item $Dispose$ вызывается у $HttpMessageHandler$ только в случае $disposeHandler = true$.
\end{itemize}
\end{frame}

\begin{frame}[fragile]
\frametitle{IHttpClientFactory}
\begin{lstlisting}
public static HttpClient CreateClient(
  this IHttpClientFactory factory)
{
  return factory
    .CreateClient(DefaultName);
}
\end{lstlisting}
\end{frame}

\begin{frame}[fragile]
\frametitle{DefaultHttpClientFactory}
\begin{lstlisting}
public HttpClient CreateClient(string name)
{
  HttpClient httpClient = new 
    HttpClient(this.CreateHandler(name), false);
  return httpClient;
}
\end{lstlisting}
\end{frame}

\begin{frame}[fragile]
\frametitle{DefaultHttpClientFactory}
\begin{lstlisting}
public HttpMessageHandler CreateHandler(
  string name)
{
  ActiveHandlerTrackingEntry entry = 
    this._activeHandlers.GetOrAdd(name, this._entryFactory).Value;
  this.StartHandlerEntryTimer(entry);
  return (HttpMessageHandler) entry.Handler;
}
\end{lstlisting}
\end{frame}

\begin{frame}
\frametitle{Создание с помощью HttpClientFactory}
\begin{itemize}
	\item $Dispose$ не имеет эффекта;
	\item $HttpClientFactory$ переиспользует $Handler$;
	\item $HttpClientFactory$ выставляет таймер для $Handler$.
\end{itemize}
\end{frame}

\begin{frame}
\frametitle{Выводы}
\begin{itemize}
	\item Named clients, Typed client, Refit;
	\item Переиспользуют $HttpMessageHandler$;
	\item Закрывают $HttpMessageHandler$ спустя время;
	\item $HttpClient$ не $IDisposable$.
\end{itemize}
\end{frame}

\section{Дополнительные улучшения в .NET Core 2.1}
\begin{frame}
\frametitle{\href{https://docs.microsoft.com/en-us/dotnet/api/system.net.http.delegatinghandler?view=netcore-2.2}{DelegatingHandler}}
\begin{itemize}
\item $DelegatingHandler$ позволяют создать цепочку обработки исходящих запросов;
\item Схоже с middleware в ASP.NET Core;
\item Функциональность была, но с $IHttpClientFactory$ стало проще использовать.
\end{itemize}
\end{frame}

\begin{frame}[fragile]
\frametitle{Создание DelegatingHandler}
\begin{lstlisting}
public class SomeHandler : DelegatingHandler

override SendAsync(...)
       
var response = await base.SendAsync(request, cancellationToken);        
\end{lstlisting}
\end{frame}

\begin{frame}[fragile]
\frametitle{Регистрация DelegatingHandler}
\begin{lstlisting}
services.AddHttpClient("some-site")
  //first
  .AddHttpMessageHandler<OutsideHandler>()
  //second
  .AddHttpMessageHandler<InsideHandler>()
\end{lstlisting}
\end{frame}

\begin{frame}
\frametitle{\href{https://github.com/App-vNext/Polly}{Polly}}
Библиотека Polly для обработки ошибок (\href{https://github.com/App-vNext/Polly}{https://github.com/App-vNext/Polly})
\begin{figure}
\includegraphics[scale=0.4]{polly}
\end{figure}
\end{frame}

\begin{frame}
\frametitle{\href{https://github.com/App-vNext/Polly}{Polly}}
\begin{itemize}
\item Подходит не только для $HttpClient$;
\item Содержит различные политики: Retry, Circuit Breaker, Timeout, \ldots
\item Необходимо установить \href{https://www.nuget.org/packages/Microsoft.Extensions.Http.Polly/}{$Microsoft.Extensions.Http.Polly$}.
\end{itemize}
\end{frame}

\begin{frame}[fragile]
\frametitle{Добавление политик}
Обработка всех ответов со статус кодами 5xx и 408
\newline
\begin{lstlisting}
services.AddHttpClient("some-site")
  .AddTransientHttpErrorPolicy(p => p.RetryAsync(3))
  .AddTransientHttpErrorPolicy(
    p => p.CircuitBreakerAsync(5, TimeSpan.FromSeconds(30)));
\end{lstlisting}
\end{frame}

\begin{frame}[fragile]
\frametitle{Настройка внутреннего \href{https://docs.microsoft.com/en-us/dotnet/api/system.net.http.socketshttphandler?view=netcore-2.2}{HttpMessageHandler}}
\begin{lstlisting}
services.AddHttpClient("some-site")
  .ConfigurePrimaryHttpMessageHandler(() =>
  {
    return new SocketsHttpHandler()
    {
      AutomaticDecompression = DecompressionMethods.GZip
    };
  });
\end{lstlisting}
\end{frame}

\begin{frame}[fragile]
\frametitle{Настройка внутреннего \href{https://docs.microsoft.com/en-us/dotnet/api/system.net.http.socketshttphandler?view=netcore-2.2}{HttpMessageHandler}}
\begin{itemize}
\item $AllowAutoRedirect$;
\item $AutomaticDecompression$;
\item $MaxAutomaticRedirections$;
\item $MaxResponseHeadersLength$;
\item \ldots
\end{itemize}
\end{frame}

\begin{frame}[fragile]
\frametitle{Время жизни \href{https://docs.microsoft.com/en-us/dotnet/api/system.net.http.socketshttphandler?view=netcore-2.2}{HttpMessageHandler}}
\begin{lstlisting}
services.AddHttpClient("some-site")
  .SetHandlerLifetime(TimeSpan.FromMinutes(5));
\end{lstlisting}
\end{frame}

\begin{frame}
\frametitle{Выводы}
\begin{itemize}
	\item Добавились методы для удобной настройки $HttpClient$.
\end{itemize}
\end{frame}


\section{HttpRequestMessage и HttpResponseMessage}
\begin{frame}
\frametitle{\href{https://docs.microsoft.com/en-us/dotnet/api/system.net.http.httprequestmessage?view=netcore-2.2}{HttpRequestMessage}}
Представляет сообщение HTTP-запроса. 
\begin{itemize}
	\item $HttpMethod\,method$;
	\item $Uri\,requestUri$;
	\item $HttpRequestHeaders\,headers$;
	\item $Version\,version$, значение по умолчанию $Version(2, 0)$;
	\item $HttpContent\,content$, который является $IDisposable$;
\end{itemize}
\end{frame}

\begin{frame}
\frametitle{Диспозить или нет?}
Зависит от $HttpContent$. Чаще всего это $StringContent$ $\Rightarrow$ можно не диспозить.
\end{frame}

\begin{frame}
\frametitle{\href{https://docs.microsoft.com/en-us/dotnet/api/system.net.http.httpresponsemessage?view=netcore-2.2}{HttpResponseMessage}}
Представляет ответное сообщение HTTP.
\begin{itemize}
	\item $HttpStatusCode\,statusCode$ (есть проверка $value > 0$ и $value < 999$);
	\item $HttpResponseHeaders\,headers$;
	\item $string\,reasonPhrase$
	\item $HttpRequestMessage\,requestMessage$
	\item $Version\,version$, значение по умолчанию $Version(1, 1)$;
	\item $HttpContent\,content$, который является $IDisposable$;
\end{itemize}
\end{frame}

\begin{frame}
\frametitle{Диспозить или нет?}
\begin{itemize}
	\item Для $GetAsync$ и $SendAsync$;
	\item $HttpCompletionOption$: $ResponseContentRead$, $ResponseHeadersRead$;
	\item Если $ResponseContentRead$, то данные сохраняются в $MemoryStream$ $\Rightarrow$ можно без диспоза;
	\item Иначе стоит диспозить. 
\end{itemize}
\end{frame}

\begin{frame}[fragile]
\frametitle{Антипаттерн чтения в строку}
\begin{lstlisting}
var response = await client.GetAsync("/get-some-data");
var str = await response.Content.ReadAsStringAsync();
var someData = JsonConvert.DeserializeObject<SomeData>(str);
return someData;
\end{lstlisting}
\end{frame}

\begin{frame}[fragile]
\frametitle{Десериализуем из stream}
\begin{lstlisting}
var srz = new JsonSerializer();
var response = await client.GetAsync("/get-some-data");
var stream = await response.Content.ReadAsStreamAsync();
using (var sr = new StreamReader(stream))
using (var jsonReader = new JsonTextReader(sr))
{
  return srz.Deserialize<SomeData>(jsonReader);
}
\end{lstlisting}
\end{frame}

\begin{frame}[fragile]
\frametitle{Десериализуем из stream в .net core 3.0}
\begin{lstlisting}
var response = await client.GetAsync("/get-some-data");
var stream = await response.Content.ReadAsStreamAsync();
var someDate = await JsonSerializer.DeserializeAsync<SomeData>(stream);
return someDate;
\end{lstlisting}
\end{frame}

\begin{frame}
\frametitle{Выводы}
\begin{itemize}
	\item $HttpRequestMessage$ и $HttpResponseMessage$ иногда можно не диспозить;
	\item Лучше не использовать промежуточную строку при десериализации.
\end{itemize}
\end{frame}

\section{Новое в .NET Core 3.0}
\begin{frame}[fragile]
\frametitle{Поддержка HTTP/2}
\begin{lstlisting}
using (var request = 
  new HttpRequestMessage(HttpMethod.Get, "/") 
    { Version = new Version(2, 0) })
\end{lstlisting}
\end{frame}

\begin{frame}[fragile]
\frametitle{Поддержка HTTP/2}
\begin{lstlisting}
var client = new HttpClient()
{
  BaseAddress = new Uri("https://localhost:80"),
  DefaultRequestVersion = new Version(2, 0)
};
\end{lstlisting}
\end{frame}

\begin{frame}[fragile]
\frametitle{Регистрация gRPC Client}
Схожий шаблон с $HttpClient$:
\newline
\begin{lstlisting}
services.AddGrpcClient<GreeterClient>(options =>
  {
    options.BaseAddress = new Uri("https://localhost:5001");
  });
\end{lstlisting}
\end{frame}

\begin{frame}
\frametitle{Ссылки}
\begin{itemize}
	\item \href{Слайды и примеры https://github.com/rafaelldi/SpbDotNet-HttpClient}{https://github.com/rafaelldi/SpbDotNet-HttpClient};
	\item \href{https://aspnetmonsters.com/2016/08/2016-08-27-httpclientwrong/}{You're using HttpClient wrong and it is destabilizing your software};
	\item \href{https://byterot.blogspot.com/2016/07/singleton-httpclient-dns.html}{Singleton HttpClient? Beware of this serious behaviour and how to fix it};
	\item \href{https://nima-ara-blog.azurewebsites.net/beware-of-the-net-httpclient/}{Beware of the .NET HttpClient};
	\item \href{https://habr.com/en/post/424873/}{Подводные камни HttpClient в .NET};
	\item \href{https://docs.microsoft.com/en-us/aspnet/core/fundamentals/http-requests?view=aspnetcore-2.2}{Make HTTP requests using IHttpClientFactory in ASP.NET Core}.
\end{itemize}
\end{frame}

\begin{frame}
\frametitle{Контакты}
\begin{itemize}
	\item Email rival.abdrakhamnov@gmail.com;
	\item Блог \href{https://arcadeprogramming.com/}{https://arcadeprogramming.com/}.
\end{itemize}
\end{frame}

\end{document}
